%----------------------------------------------------------------------------------------
%	PAQUETES GENERALES
%----------------------------------------------------------------------------------------
%----------------------------------------------------------------------------------------
%	PAQUETES GENERALES
%----------------------------------------------------------------------------------------
\usepackage[spanish,es-noshorthands]{babel} % Configuración para palabras en Español
\usepackage{microtype} 						% Proporciona características avanzadas de microtipografía
\usepackage{XCharter} 						% Proporciona una familia de fuentes basada en la fuente Charter
\usepackage{lettrine} 						% Paquete para acentuar la primer letra de un texto (lettrine)
\usepackage{amsmath,amsfonts,amsthm} 		% Paquetes para matemáticas
\usepackage{graphicx} 						% Requerido para agregar imágenes
\usepackage{svg}
\usepackage{float}							% Requerido para el posicionamiento flotante
\usepackage{tcolorbox}						% Permite hacer recuadros para definiciones
\usepackage{multicol}						% Multicolumnas
\usepackage[hidelinks]{hyperref}			% Este paquete se utiliza para manejar enlaces dentro del documento
\usepackage{color}							% Permite colocar colores a los textos
\usepackage{circuitikz}						% Permite crear circuitos eléctricos




%----------------------------------------------------------------------------------------
%	MARGINS AND SPACING
%----------------------------------------------------------------------------------------
\usepackage{geometry}
\geometry{
	top=1cm,
	bottom=1cm,
	left=1cm,
	right=1cm,
	includehead,
	includefoot,
	%showframe, % Descomentar para visualizar los limites
}
\setlength{\columnsep}{5mm} 				% Separacion entre columnas
% \setlength{\columnseprule}{0.1mm} 		% Linea entre columnas
\usepackage{titlesec}						% Permite modificar los espacios de los titulos
\titlespacing{\section}{0pt}{10pt}{5pt}		% Izquierda del titulo, Antes y despues del mismo



%----------------------------------------------------------------------------------------
%	ENCABEZADO Y PIE DE PAGINA
%----------------------------------------------------------------------------------------
\usepackage{fancyhdr}	% Permite modificar los Encabezados y Pie de Pagina
\usepackage{lastpage} 	% Used to determine the number of pages in the document (for "Page X of Total")
\pagestyle{fancy}
\fancyhf{}
\renewcommand{\headrulewidth}{1pt}	% Separacion post-header
\renewcommand{\footrulewidth}{1pt}	% Separacion pre-footer
\fancyfoot[OR, EL]{Pagina \thepage/\pageref*{LastPage}}
\fancyfoot[OL, ER]{\autor}
\fancyhead[OR, EL]{\today}
\fancyhead[OL, ER]{\tituloHeader}


%----------------------------------------------------------------------------------------
%	FORMATO DEL TITULO Y DEL RESUMEN COMO COMANDO
%----------------------------------------------------------------------------------------
\fancypagestyle{firstpage}{ 		% Solo para la pagina que tenga "\thispagestyle{firstpage}"
	\fancyhead{}
	\renewcommand{\headrulewidth}{0pt}
}
\newcommand{\TituloyResumen}[2]{
	\thispagestyle{firstpage}
	\vspace*{-1.5cm}

	% TITULO
	\begin{center}
		{\huge\bfseries \titulo}\vspace{5mm}\break
		{\large \autor}\break
		{\establecimiento}
	\end{center}

	% RESUMEN
	\begin{center}
		\rule{\linewidth}{0.1mm}
		\begin{flushleft}
			\begin{Large}
				\textbf{Resumen}
			\end{Large}\break
			#1
			\vspace{3mm}\break
			\textit{\textbf{Palabras Claves}}: #2
		\end{flushleft}
		\vspace{-\baselineskip}
		\rule{\linewidth}{0.1mm}
	\end{center}

	% FECHA
	\begin{center}
		\textit{\today}
	\end{center}
}



%----------------------------------------------------------------------------------------
%	TEXTO EN GENERAL
%----------------------------------------------------------------------------------------
\usepackage{parskip}							% Al usarlo, elimina todas las sangrias.
\setlength{\parindent}{0pt} 					% Elimina la sangría
\setlength{\parskip}{6pt plus 1pt minus 1pt} 	% Separacion base: 6pt, Maximo adicional 2pt y minimo adicional 1pt
\setlength{\headheight}{13.6pt}



%----------------------------------------------------------------------------------------
%	CONFIGURACION DEL CUERPO (TABLAS Y FIGURAS)
%----------------------------------------------------------------------------------------
\usepackage{booktabs}		% Para centrar las tablas
\definecolor{GrayCaptions}{rgb}{0.5,0.5,0.5}
\usepackage[
	font={footnotesize, color=GrayCaptions},   % Aplica el color a los captions
	figurename=Imagen,    % Cambia el nombre de las figuras a "Imagen"
	tablename=Tabla,      % Cambia el nombre de las tablas a "Tabla"
	labelfont={it}        % Aplica itálico a las etiquetas
]{caption}
\setlength{\abovecaptionskip}{5pt plus 2pt minus 2pt} % Separacion de los 'captions' con el objeto
\setlength{\belowcaptionskip}{-12pt plus 2pt minus 2pt} % Separacion de los 'captions' con el texto que continua.




%----------------------------------------------------------------------------------------
%	BIBLIOGRAFIA (Para deshabilitar, comentar lo siguiente)
%----------------------------------------------------------------------------------------
\usepackage{csquotes}
\usepackage[
backend=biber,
style=apa,
sortcites,
url=true
]{biblatex}
\addbibresource{Bibliografia.bib}



